\documentclass[14pt]{article}

\usepackage{preamble}
\usepackage{ppc}

\AtBeginDocument{\addtocontents{toc}{\protect\thispagestyle{fancy}}}

%Mise en page en mode fancy : en-têtes et pieds de pages puis définition des en-têtes et pieds de pages%
\pagestyle{fancy}
\cfoot[\thepage]{\thepage}
\begin{document}


%Trait en bas et en haut de la page (entre en-tête et texte et texte et pied de page)%
\renewcommand{\footrulewidth}{0.4pt}
\renewcommand{\headrulewidth}{0.4pt}

\vspace{0cm}              %espace en hauteur (hspace pour la largeur)%
\setstretch{1.4}

%DEBUT DU DOCUMENT%

\import{./}{Front_page.tex}

\clearpage
\thispagestyle{fancy}
\selectlanguage{french}
\tableofcontents
\thispagestyle{fancy}      %Mise en page de la 1ère page en mode fancy%

\clearpage

\section{Choix d'implémentation}

Nous avons implémenté notre solveur en C++.

\subsection{Structure des données}

%TODO%
On présente dans cette section les choix de représentation mémoire d'un CSP binaire, et les raisons justifiant ces choix.

\subsection{Représentation fichier}

Afin de standardiser les entrées fournies à notre solveur, on a proposé un format standardisé de fichier d'entrée. Le solveur prend ainsi en entrée un fichier .json dont la structure est décrite ci-dessous:

\begin{verbatim}
	{   "nVar": 2,
	    "Domaines": [{   "vars": [0,1], "values": [0,1]    }],
	    "Constraints": [{   "vars": [[1,0]], "allowed": [[1,1]]    }]}
\end{verbatim}

C'est une description en extension des contraintes binaires de notre CSP, dans laquelle on s'autorise toutefois à réunir les variables qui possèdent les mêmes domaines, ainsi que les couples de variables possédant les mêmes contraintes. Une telle description en extension du problème atteint rapidement ses limites en terme de taille mémoire. Ci-dessous le nombre d'octets nécessaire pour représenter n-queens pour différentes valeurs de n.

\begin{longtable}{|c|c|}
	\hline
	$n$ & Taille mémoire (Mo) \\
	\hline
	\endfirsthead
	
	$n$ & Taille mémoire (Mo) \\
	\hline
	\endhead
	
	\hline
	\endfoot
	
	\hline
	\endlastfoot
	
	10  & 0.1 \\
	25  & 1.2 \\
	50  & 22.2 \\
	100  & 377.2 \\
\end{longtable}

Lorsqu'on double la valeur de n, on augmente la taille mémoire par un facteur 20 environ. Il devient rapidement déraisonnable de souhaiter représenter de tels problèmes en extension pour de plus grande valeurs de n. En pratique on se limitera à n=100 pour les instances de n-reines considérées plus bas.


\subsection{Implémentation d'algorithmes}

%TODO%
On présente dans cette section quelques choix originaux d'implémentation des algorithmes de résolution d'un CSP, et les raisons justifiant ces choix.

\clearpage

\section{Évaluation du solveur}

\subsection{Évaluation du choix de la méthode de résolution}

\subsubsection{Présentation des méthodes}

%TODO%
On décrit ici les différentes méthodes abordées en cours pour la résolution d'un CSP binaire implémentées dans notre solveur.

\subsubsection{Évaluation sur n-reines}

%TODO%
On présente ci-dessous une évaluation des performances de notre solveur quant au temps de résolution nécessaire pour exhiber une solution du problème n-reines.

\subsubsection{Évaluation sur le problème de k-coloriage}

%TODO%
On présente ci-dessous le temps de résolution nécessaire à l'obtention d'une solution pour des problèmes de k-coloriage, choisis par ordre de difficulté croissante.

\subsection{Évaluation du choix de sélection de variable}

\subsubsection{Présentation des heuristiques implémentées}

%TODO%
On décrit dans cette section les différentes heuristiques de choix de variable utilisables par le solveur.

\subsubsection{Évaluation sur n-reines}

%TODO%
On met en évidence ici le fait qu'un choix de variables aléatoires permet d'obtenir une solution bien plus rapidement qu'aucune autre heuristique.

\subsubsection{Évaluation sur le problème de k-coloriage}

%TODO%
On tente d'observer si la même observation s'applique au problème de coloriage d'un graphe.

% A envisager potentiellement? Trouver un type de problème pour lequel l'heuristique de choix aléatoire de variables n'est pas la meilleure.

\end{document}
