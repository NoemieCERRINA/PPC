\documentclass[14pt]{article}

\usepackage{preamble}
\usepackage{ppc}

\AtBeginDocument{\addtocontents{toc}{\protect\thispagestyle{fancy}}}

%Mise en page en mode fancy : en-têtes et pieds de pages puis définition des en-têtes et pieds de pages%
\pagestyle{fancy}
\cfoot[\thepage]{\thepage}
\begin{document}


%Trait en bas et en haut de la page (entre en-tête et texte et texte et pied de page)%
\renewcommand{\footrulewidth}{0.4pt}
\renewcommand{\headrulewidth}{0.4pt}

\vspace{0cm}              %espace en hauteur (hspace pour la largeur)%
\setstretch{1.4}

%DEBUT DU DOCUMENT%

\import{./}{Front_page.tex}

\clearpage
\thispagestyle{fancy}
\selectlanguage{french}
\tableofcontents
\thispagestyle{fancy}      %Mise en page de la 1ère page en mode fancy%

\clearpage

\section{Choix d'implémentation}

Nous avons implémenté notre solveur en C++.

\subsection{Structure des données}

\subsubsection{Structure des contraintes} \label{Structure des contraintes}
Chaque contrainte est modélisée par un objet de classe \texttt{Constraint} qui contient les informations suivantes:
\begin{itemize}
    \item Les indices des variables concernées par la contrainte. Notre solveur ne prenant que des CSP binaires en entrée, il s'agit donc de deux variables entières et ordonnées.
    \item Une liste des paires de valeurs autorisées pour ces variables, sous forme de vecteur de paires d'entiers.
    \item Un dictionnaire de clés où chaque clé correspond à une paire de valeurs autorisées dans le domaine des variables considérées.
\end{itemize}

La liste des paires de valeurs autorisées, implémentée initialement de manière naïve ne permettait la vérification rapide du support d'une variable par une autre. D'où la construction du dictionnaire qui permet quant à lui l'accès en temps constant à la réponse à la question "La paire de variables $x_i$ et $x_j$ peut-elle prendre les valeurs $(v_i,v_j)$ en respectant la contrainte $C(x_i,x_j)$?"

\subsubsection{Structure du CSP}

Afin de résoudre des CSP binaires, nous les avons représenté dans notre solveur à l'aide d'une classe CSP contenant les informations suivantes:

\begin{itemize}
    \item Le nombre de variables du problème sous la forme d'un entier.
    \item Une liste des des domaines de chaque variable. Chaque domaine est un entier correspondant aux valeurs autorisées pour chaque variable.
    \item Un vecteur de contraintes, où chaque contrainte est un objet de la classe contraintes décrite dans la section \ref{Structure des contraintes}.
    \item Une matrice des contraintes, où chaque élément en $(i,j)$ est soit un pointeur nul s'il n'existe pas de contrainte entre la variable $x_i$ et $x_j$, soit un pointeur vers la contrainte correspondante. Cette matrice permet l'accès en temps constant à la contrainte entre deux variables $x_i$ et $x_j$ données.
\end{itemize}

\subsection{Représentation fichier}

%TODO%
On présente dans cette section le format de données d'entrée fournies au solveur, et une critique quant à l'utilisation mémoire d'un tel format.

\subsection{Implémentation d'algorithmes}

%TODO%
On présente dans cette section quelques choix originaux d'implémentation des algorithmes de résolution d'un CSP, et les raisons justifiant ces choix.

\section{Évaluation du solveur}

\subsection{Évaluation du choix de la méthode de résolution}

\subsubsection{Présentation des méthodes}

%TODO%
On décrit ici les différentes méthodes abordées en cours pour la résolution d'un CSP binaire implémentées dans notre solveur.

\subsubsection{Évaluation sur n-reines}

%TODO%
On présente ci-dessous une évaluation des performances de notre solveur quant au temps de résolution nécessaire pour exhiber une solution du problème n-reines.

\subsubsection{Évaluation sur le problème de k-coloriage}

%TODO%
On présente ci-dessous le temps de résolution nécessaire à l'obtention d'une solution pour des problèmes de k-coloriage, choisis par ordre de difficulté croissante.

\subsection{Évaluation du choix de sélection de variable}

\subsubsection{Présentation des heuristiques implémentées}

%TODO%
On décrit dans cette section les différentes heuristiques de choix de variable utilisables par le solveur.

\subsubsection{Évaluation sur n-reines}

%TODO%
On met en évidence ici le fait qu'un choix de variables aléatoires permet d'obtenir une solution bien plus rapidement qu'aucune autre heuristique.

\subsubsection{Évaluation sur le problème de k-coloriage}

%TODO%
On tente d'observer si la même observation s'applique au problème de coloriage d'un graphe.

% A envisager potentiellement? Trouver un type de problème pour lequel l'heuristique de choix aléatoire de variables n'est pas la meilleure.

\end{document}
