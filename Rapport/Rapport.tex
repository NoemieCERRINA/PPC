\documentclass[14pt]{article}

\usepackage{preamble}
\usepackage{ppc}

\AtBeginDocument{\addtocontents{toc}{\protect\thispagestyle{fancy}}}

%Mise en page en mode fancy : en-têtes et pieds de pages puis définition des en-têtes et pieds de pages%
\pagestyle{fancy}
\cfoot[\thepage]{\thepage}
\begin{document}


%Trait en bas et en haut de la page (entre en-tête et texte et texte et pied de page)%
\renewcommand{\footrulewidth}{0.4pt}
\renewcommand{\headrulewidth}{0.4pt}

\vspace{0cm}              %espace en hauteur (hspace pour la largeur)%
\setstretch{1.4}

%DEBUT DU DOCUMENT%

\import{./}{Front_page.tex}

\clearpage
\thispagestyle{fancy}
\selectlanguage{french}
\tableofcontents
\thispagestyle{fancy}      %Mise en page de la 1ère page en mode fancy%

\clearpage

\section{Choix d'implémentation}

Nous avons implémenté notre solveur en C++.

\subsection{Structure des données}

%TODO%
On présente dans cette section les choix de représentation mémoire d'un CSP binaire, et les raisons justifiant ces choix.

\subsection{Représentation fichier}

%TODO%
On présente dans cette section le format de données d'entrée fournies au solveur, et une critique quant à l'utilisation mémoire d'un tel format.

\subsection{Implémentation d'algorithmes}

%TODO%
On présente dans cette section quelques choix originaux d'implémentation des algorithmes de résolution d'un CSP, et les raisons justifiant ces choix.

\section{Évaluation du solveur}

\subsection{Évaluation du choix de la méthode de résolution}

\subsubsection{Présentation des méthodes}

%TODO%
On décrit ici les différentes méthodes abordées en cours pour la résolution d'un CSP binaire implémentées dans notre solveur.

\subsubsection{Évaluation sur n-reines}

%TODO%
On présente ci-dessous une évaluation des performances de notre solveur quant au temps de résolution nécessaire pour exhiber une solution du problème n-reines.

\subsubsection{Évaluation sur le problème de k-coloriage}

%TODO%
On présente ci-dessous le temps de résolution nécessaire à l'obtention d'une solution pour des problèmes de k-coloriage, choisis par ordre de difficulté croissante.

\subsection{Évaluation du choix de sélection de variable}

\subsubsection{Présentation des heuristiques implémentées}

%TODO%
On décrit dans cette section les différentes heuristiques de choix de variable utilisables par le solveur.

\subsubsection{Évaluation sur n-reines}

%TODO%
On met en évidence ici le fait qu'un choix de variables aléatoires permet d'obtenir une solution bien plus rapidement qu'aucune autre heuristique.

\subsubsection{Évaluation sur le problème de k-coloriage}

%TODO%
On tente d'observer si la même observation s'applique au problème de coloriage d'un graphe.

% A envisager potentiellement? Trouver un type de problème pour lequel l'heuristique de choix aléatoire de variables n'est pas la meilleure.

\end{document}
